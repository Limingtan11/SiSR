参考资料
\begin{itemize}
\item  高性能计算之并行编程技术—— MPI并行程序设计
\end{itemize}
\chapter{背景知识}
\section{并行语言}
并行程序是通过并行语言来表达的,并行语言的产生主要有三种方式:
\begin{itemize}
\item 设计全新的并行语言\\
设计一种全新的并行语言的优点是可以完全摆脱串行语言的束缚, 从语言成分上直接支持并行。 这样就可以使并行程序的书写更方便, 更自然, 相应的并行程序也更容易在并行机上实现。 但是, 由于并行计算至今还没有象串行计算那样统一的冯•诺伊曼模型可供遵循,因此,并行机、并行模型、 并行算法和并行语言的设计和开发千差万别 ,没有一个统一的标准。虽然有多种多样全新的并行语言出现,但至今还没有任何一种新出现的并行语言, 成为普遍接受的标准。 设计全新的并行语言, 实现起来难度和工作量都很大。
\item 扩展原来的串行语言的语法成分使它支持并行特征\\
一种重要的对串行语言的扩充方式就是标注, 即将对串行语言的并行扩充作为原来串行语言的注释,对于这样的并行程序, 若用原来的串行编译器来编译, 标注的并行扩充部分将不起作用, 仍将该程序作为一般的串行程序处理, 若使用扩充后的并行编译器来编译, 则该并行编译器就会根据标注的要求, 将原来串行执行的部分转化为并行执行。对串行语言的并行扩充, 相对于设计全新的并行语言, 显然难度有所降低, 但需要重新开发编译器, 使它能够支持扩充的并行部分 。一般地, 这种新的编译器往往和运行时支持的并行库相结合。如openMP(Open Multi-Processing)。
\item 不改变串行语言 仅为串行语言提供可调用的并行库\\
仅仅提供并行库,是一种对原来的串行程序设计改动最小的并行化方法。 这样, 原来的串行编译器也能够使用 ,不需要任何修改 ,编程者只需要在原来的串行程序中加入对并行库的调用 就可以实现并行程序设计。MPI并行程序设计, 就属于这种方式。
\end{itemize}  
对于这三种并行语言的实现方法, 目前最常使用的是第二种和第三种方法 特别是第三
种方法。










