\chapter{Linux命令}
\section{命令行环境及命令格式}
(1)命令行环境的含义

例如:\verb*|phileas@phileas-Aspire:~/Downloads$|

\verb|phileas|是用户名,\verb|phileas-Aspire|是主机名,\verb|~/Downloads|是当前目录。对普通用户来说,提示符显示为“\verb|$|”,管理员root用户提示符为“\verb|#|”。


(2)Linux的命令格式

\verb*|$命令名 [-选项] 操作对象|

需要注意的是,命令名、选项或者操作对象之间都是至少要有一个空格,多个空格没有关系。选项和操作对象可以有多个,尤其是选项不是必需的,但也可以有多个。当有多个选项时,可以写在一起,前面通常要有一个“-”。操作对象也可以有0个、1个或多个。

由于Linux基本是C语言编写的,所以无论命令名或选项,Linux都是严格区分大小写。

\subsection{通配符}
在bash下,可以使用*来匹配零个或多个字符,而用?匹配一个字符。



\subsection{管道符}
管道符``|",就是把前面的命令运行的结果丢给后面的命令。


\subsection{作业控制}
当运行一个进程时,你可以使它暂停(按Ctrl+z),然后使用fg命令恢复它,利用bg命令使他到后台运行,你也可以使它终止(按Ctrl+c)。







\section{获得帮助的命令}
\subsection{help}
Linux命令可以分为内部命令和外部命令,所谓内部命令就是由Linux默认Shell-bash提供的命令,而非bash提供的命令就是外部命令。

对于内部命令,大家可以使用help命令来获得帮助,例如要获得cd命令的帮助可以使用如下命令来查询:

\verb*|$help cd|

其实,日常使用中大家碰到的绝大多数命令都是外部命令,所以不必刻意区分内部命令还是外部命令,只需要大家有这个概念就好。对于外部命令浏览其帮助文档,则用man命令。


\subsection{man}
man命令\footnote{manual的缩写,linux中的命令很喜欢用缩写}可以向用户提供快速的在线帮助,Linux包括了一套完整的man和man帮助文档。这些man帮助文档涉及Linux系统的命令、程序、配置文件和程序库的功能等说明。一直以来,这些文档都是某些操作系统书籍的主要资料来源,自从有了互联网和图形用户界面后,直接使用这些文档的人变少了,但它仍是很重要的信息来源。操作系统安装好后,这些文档就可以马上使用。

man的帮助文件有很多个,通常以章分类,存放在\verb|/usr/share/man|目录中,每章讨论一项专题,分为man1、man2、man3等子目录。章的划分符合AT\&T Unix的文档结构,继承了其方便使用的特点,并且保持向后兼容(Unix经常保持向后兼容)。大多数的用户命令可以在man1中找到,如果要使用其他手册,请在man和所查找命令之间加上具体手册的编号,如:

\verb*|$man password|

\verb*|$man 5 password|

第一个man命令查找的是passwd命令的manpages,而第二个man命令查找的则是/ect/passwd文件的manpages,其中5表示配置文件手册类型,而非默认的可执行程序类型。


\section{文件和目录管理命令}
\subsection{cd——改变当前路径的命令}
\verb*|$cd 路径|

特殊目录的表示方法及含义
\begin{itemize}
\item \verb|.| 代表当前目录
\item \verb|..| 代表上层目录
\item \verb|~|代表当前登录用户的宿主目录
\item \verb|~用户名| 代表进入\verb|~|后用户的宿主目录
\item \verb|-| 代表前一目录,即进入当前目录之前操作的目录
\item \textbf{直接在命令行中输入cd命令而不加任何参数,可以马上回到用户的主目录(home)}
\end{itemize}


\subsection{pwd——查看当前路径命令}
\verb*|pwd|,Print Working Directory的缩写。

如果有时在操作过程中忘记了当前的路径,则可以通过次命令来查看路径。

除了pwd命令,还有一个很有用的命令,那就是tree命令,它以更为直观的树状图的形式显示目录之间的关系。但是Ubuntu默认没有安装tree,需要手动安装,具体安装方法如下:

\verb*|$sudo aptitude install tree|


\subsection{ls——列出文件清单命令}
ls 命令能够显示目录内容和查看文件详细信息,ls意为“list directory contents”。 ls 命令的执行方式为:
\verb*|$ls -[选项] [文件名或者目录名]|
\footnote{参考资料:\url{https://www.cnblogs.com/jichi/p/11186974.html} }

ls 命令选项:
\begin{itemize}
\item 不接任何选项和目录,显示当前目录下没有隐藏的文件和目录名
\item  -a: 列出所有文件,包括隐藏文件也显示出来
\item  -l: 显示没有隐藏的文件和文件夹的详细信息
\item -al:显示当前目录下的所有文件和目录的详细信息
\item -R: 列出当前目录下的所有内容,并且将子目录的内容也一起列出来
\item -d: 仅列出目录本身,而不显示当前目录下的内容
\end{itemize}

文件详细信息详解:
\begin{itemize}
\item 假设执行\verb|ls -l|,有:\\
\verb|drwxr-xr-x 1 xuanleng xuanleng     4096 Dec  2 05:44 rpm|

\item 第1列:代表文件的类型。我们常见的是d和-。d代表是目录文件。-代表是普通文件。其他不常见的有。l代表链接文件,b代表块设备。c代表字符设备文件。

\item 第2-10列:代表该文件的权限。三个为一组。第一组代表文件所有者的权限,第二组代表同用户组的权限,第三组代表其他用户非本用户组的权限。每组权限中的rwx,分别代表读,写,可执行的意思。

\item 第11列数字,1,代表有多少文件名连接到此节点。
每个文件都会将它的权限与属性记录到文件系统的i-node中,不过我们使用的目录树却是使用文件名来记录,因此每个文件名就会连接到一个i-node,这个属性记录的就是有多少不同的文件名连接到相同的一个i-node号码。

\item 第12列的root:所属者用户名。

\item 第13列的root:所属用户组。

\item 第14列:容量大小,默认为B。

\item 后面为创建或修改日期,最后为文件名。
\end{itemize}

命令组合使用:
\begin{itemize}
\item 查看当前目录下的目录个数:\verb?ls -l |grep "^d"|wc -l?
\end{itemize}



\subsection{tar——解压、压缩命令}
\verb|tar -zxvf 文件名|:将文件直接解压到当前目录下



\subsection{chmod——修改权限命令}
每一文件或目录的访问权限都有三组,每组用三位表示,分别为文件属主的读、写和执行权限;与属主同组的用户的读、写和执行权限;系统中其他用户的读、写和执行权限。
-代表空许可,r代表只读,w代表只写,x代表可执行。\footnote{参考资料:\url{https://www.cnblogs.com/jichi/p/11186974.html} }

chmod 命令用于改变文件或目录的访问权限。该命令有两种写法:1、包含字母和操作符的方式。2、包含数字的设定方式。

\subsubsection{文字设定法}
一般形式为
\verb*|chmod [who] [+\-\=] [mode] 文件或目录名|

命令中各选项的含义:
\begin{itemize}
\item 操作对象who可是下述字母中的任一个或者它们的组合:
\begin{itemize}
\item u 表示“用户(user)”,即文件或目录的所有者。  
\item g 表示“同组(group)用户”,即与文件属主有相同组ID的所有用户。  
\item o 表示“其他(others)用户”。  
\item a 表示“所有(all)用户”。它是系统默认值。 
\end{itemize} 

\item 操作符号可以是:
\begin{itemize}
\item +添加某个权限。
\item -取消某个权限。  
\item =赋予给定权限并取消其他所有权限(如果有的话)。  
\end{itemize}

\item 设置mode所表示的权限可用下述字母的任意组合:  
\begin{itemize}
\item r 可读。  
\item w 可写。  
\item x 可执行。  
\item u 与文件属主拥有一样的权限。
\item g 与和文件属主同组的用户拥有一样的权限。
\item o 与其他用户拥有一样的权限。
\end{itemize}

\item 文件名:以空格分开的要改变权限的文件列表,支持通配符。

\end{itemize}

在一个命令行中可给出多个权限方式,其间用逗号隔开。

例如:chmod g+r,o+r example

使同组和其他用户对文件example 有读权限。

例如:chmod u+x startup.sh

给当前用户增加对startup.sh的执行权限。


\subsubsection{数字设定法}
chmod [mode] 文件名

我们将之前的rwx用数字进行替代。0表示没有权限,1表示可执行权限,2表示可写权限,4表示可读权限。数字之和,即为该文件的权限。使用文字的方式,有三组文字,数字即为3组数字之和。

例子

1、chmod 644 a.txt

文件属主具有读,写权限,因为6=4+2。

文件组具有读权限。

其他用户具有读权限


\subsection{chgrp——修改文件或目录所属组的命令}
改变文件或目录所属的组

chgrp [选项] group filename

- R 递归式地改变指定目录及其下的所有子目录和文件的属组。

group为用户组id或者为/etc/group中用户的用户组名。如果用户不是文件的属主或者超级用户,则不能改变。

chgrp - R jichi /etc

将etc及etc下所有文件和目录的属组都改为jichi。


\subsection{chown——修改文件或目录所属主的命令}
更改某个文件或目录的属主和属组。

chown [选项] 用户或组 文件  

- R 递归式地改变指定目录及其下的所有子目录和文件的拥有者。 
 
- v 显示chown命令所做的工作。  

例子

把文件a.txt的所有者改为jichi。

chown jichi a.txt  

把目录/b及其下的所有文件和子目录的属主改成jichi,属组改成jingdian。

chown - R jichi.jingdian /b




\subsection{mkdir——创建目录命令}
\verb*|$mkdir 目录名|

对于一次性建立多级目录的情况,可以使用\verb*|-p|参数:

\verb*|$mkdir -p /home/dir1/dir2/dir3|


\subsection{rmdir——删除空目录命令}
\verb*|$ rmdir 目录名|

例如要删除目录\verb*|/home/hxl/directory|这个空目录,则可以执行命令:

\verb*|$rmdir /home/hxl/directory|

注意:\verb*|rmdir|可以删除的是非空目录,被删的目录下不能有文件或子目录。如果只是有子目录存在,还可以用\verb*|-p|来删除;如果含有文件,那么\verb*|-p|选项也无能为力了。


\subsection{mv——移动或更名现有的文件或目录}
linux下重命名文件或文件夹的命令mv既可以重命名,又可以移动文件或文件夹.

例子:将目录A重命名为B

mv A B

例子:将/a目录移动到/b下,并重命名为c

mv /a /b/c

其实在文本模式中要重命名文件或目录的话也是很简单的,我们只需要使用mv命令就可以了,比如说我们要将一个名为abc的文件重命名为1234就可以这样来写:mv abc 1234,但是要注意的是,如果当前目录下也有个1234的文件的话,我们的这个文件是会将它覆盖的


\subsection{rm——删除文件(目录)命令}
\verb*|rm|命令能够删除一个文件或者目录。

\verb*|$rm [-选项] 文件名或者目录名|

对于Ubuntu来说,这个命令是比较危险的一个命令,因为一旦使用了这个命令删除的文件如果再进行磁盘写操作后将无法再恢复。在其他版本的一些Linux中,默认情况下给\verb*|rm|命令加上一个\verb*|-i|选项,可以在删除前对操作进行确认。

\verb*|rm|命令选项:
\begin{itemize}
\item -f:强制删除(即使设置了\verb*|-i|属性也不起作用)
\item -r:删除目录
\item -i:删除文件或目录前是否询问
\end{itemize}


\subsection{cp——复制命令}
cp命令能够复制一个文件或者生成一个不同名但是内容相同的文件。
\begin{itemize}
\item 复制文件:\verb*|$ cp 源文件名 目标路径|

\verb*|$ cp /etc/passwd /home/hxl/|:将目录\verb|/etc|下的\verb|passwd|文件复制到\verb|/home/hxl|目录中。

\item 当前目录下生成不同名但内容相同的文件:\verb*|$ cp 源文件名 目标文件名|

\verb*|$ cp passwd passwd_new|

\item 复制目录时要加上选项\verb*|-r|,例如在当前目录下复制目录a,并重命名为b:

\verb*|$ cp -r a b|
\end{itemize}


\section{查看命令}
\subsection{man——查看命令手册}
man:manual的缩写,用来查看命令的使用手册,如:\verb*|man  cp|,了解命令用法的很好用的命令。


\subsection{file——检测文件类型}
 file 是检测文件类型\footnote{文件类型就文件组织的方式,通常不同的文件类型执行不同的标准。例如我们熟知的:txt , doc , xls , pdf ...}的命令。file 命令的简单用法就是:file 文件名,例如:

\verb*|file data.txt|

data.txt: ASCII text

就告诉我们,data.txt 是一个text (即txt) 类型的文件。txt 文件所采用的编码是ascii编码体系。所以 text 是文件类型;ASCII是编码体系。

又如:

\verb*|file my.pdf|

my.pdf: PDF document, version 1.5

“PDF document” 告诉我们 , my.pdf 是pdf类型的文件。版本执行的标准是:1.5 。

像ascii , version 1.5 这些都是与文件类型密切相关的信息。如果需要更详细的信息,可以加参数:如:\verb*|file -i data.txt|。可以使用man file 查看详细用法。


\section{文本处理命令}
\subsection{awk}
\subsubsection{awk常用内置变量}


\subsubsection{awk统计命令}
1、求和
\begin{lstlisting}[language=sh]
cat data|awk '{sum+=$1} END {print "Sum = ", sum}'
\end{lstlisting}
例如:
\begin{lstlisting}[language=sh]
xuanleng@xuanleng-Alienware-15:~/Desktop$ cat 2dspectra_xyz.ods|awk '{sum+=$1} END {print "Sum = ", sum}'
Sum =  4.79883e+06
\end{lstlisting}

2、求平均
\begin{lstlisting}[language=sh]
cat data|awk '{sum+=$1} END {print "Average = ", sum/NR}'
\end{lstlisting}

3、求最大值
\begin{lstlisting}[language=sh]
cat data|awk 'BEGIN {max = 0} {if ($1>max) max=$1 fi} END {print "Max=", max}'
\end{lstlisting}

4、求最小值(min的初始值设置一个超大数即可)
\begin{lstlisting}[language=sh]
cat data|awk 'BEGIN {min = 1999999} {if ($1<min) min=$1 fi} END {print "Min=", min}'
\end{lstlisting}



\subsection{grep}
grep,可以在多个文本文件中查找指定的字符串,如:在/home/phileas/Desktop目录下搜索带字符串‘magic’的文件

\verb|grep magic /home/phileas/Desktop/*|

其中``*"是通配符。

当然也可以搜索单个文件,如:\verb*[grep 'BCL A' 3BSD.pdb[

用于搜索的特殊符号:
\begin{itemize}
\item \verb|\<| 和\verb| \>| 分别标注单词的开始与结尾。 例如:\\ 
\verb|grep 'man' |: 会匹配 ‘Batman’、‘manic’、‘man’等';\\
\verb|grep '\<man' |: 匹配‘manic’和‘man’,但不是‘Batman’;\\ 
\verb|grep '\<man\>' |: 只匹配‘man’,而不是‘Batman’或‘manic’等其他的字符串。 

\item \verb|'^'|:指匹配的字符串在行首;

\item \verb|'$'|:指匹配的字符串在行尾。
\end{itemize}


 \verb*[grep 'BCL A' 3BSD.pdb | grep "^HETATM" | grep NB | tr -s ' ' ' ' | cut -d' ' -f7,8,9 >NB.ods[ 

命令说明:
\begin{itemize}
\item \verb*| | :表示空格;
\item \verb*[grep 'BCL A' 3BSD.pdb[ :意思是提取出文件3BSD.pdb中包含字符 BCL A 的内容;
\item |:管道符,将之前命令的结果传给后面的命令;
\item \verb*[grep "^HETATM"[: \verb|'^'|,指匹配的字符串在行首;意思是提取出首行含有字符HETATM的内容;
\item \verb*|grep NB|:提取出含有字符 NB 的内容;
\item \verb*| tr -s ' ' ' ' | cut -d' ' -f7,8,9|:删除字符和空格,留下所需数据;
\item >NB.ods :>,重定向命令;将最终结果输入到 NB.ods 文件中。
\end{itemize}




\subsection{tr}
通过使用 tr,可以非常容易地实现 sed 的许多最基本功能。可以将 tr 看作为 sed 的(极其)简化的变体:它可以用一个字符来替换另一个字符,或者可以完全除去一些字符。也可以用它来除去重复字符。这就是所有 tr 所能够做的。 


tr用来从标准输入中通过替换或删除操作进行字符转换。tr主要用于删除文件中控制字符或进行字符转换。使用tr时要转换两个字符串:字符串1用于查询,字符串2用于处理各种转换。tr刚执行时,字符串1中的字符被映射到字符串2中的字符,然后转换操作开始。

\textbf{(1)最常用选项的tr命令格式}

\verb|tr -c -d -s ["string1_to_translate_from"] ["string2_to_translate_to"] < input-file|

这里:
\begin{itemize}
\item -c 用字符串1中字符集的补集替换此字符集,要求字符集为ASCII。
\item -d 删除字符串1中所有输入字符。
\item -s 删除所有重复出现字符序列,只保留第一个。即将重复出现字符串压缩为一个字符串。
\item input-file是转换文件名。虽然可以使用其他格式输入,但这种格式最常用。
\end{itemize}

\textbf{(2)字符范围}

指定字符串1或字符串2的内容时,只能使用单字符或字符串范围或列表。
\begin{itemize}
\item [a-z] a-z内的字符组成的字符串;
\item [A-Z] A-Z内的字符组成的字符串;
\item [0-9] 数字串;
\item \verb|\octal| 一个三位的八进制数,对应有效的ASCII字符;
\item [O*n] 表示字符O重复出现指定次数n,因此[O*2]匹配OO的字符串。
\end{itemize}

\textbf{(3)tr中特定控制字符的不同表达方式}

速记符含义八进制方式
\begin{itemize}
\item \verb|\a Ctrl-G  铃声\007|
\item \verb|\b Ctrl-H  退格符\010|
\item \verb|\f Ctrl-L  走行换页\014|
\item \verb|\n Ctrl-J  新行\012|
\item \verb|\r Ctrl-M  回车\015|
\item \verb|\t Ctrl-I  tab键\011|
\item \verb|\v Ctrl-X  \030|
\end{itemize}


\textbf{(4)实例}
\begin{enumerate}
\item 将文件file中出现的“abc”替换为“xyz”\\
\verb?cat file | tr "abc" "xyz" > new_file?  \textbf{注意:}这里,凡是在file中出现的“a”字母,都替换成“x”字母,“b”字母替换为“y”字母,“c”字母替换为“z”字母。而不是将字符串“abc”替换为字符串“xyz”。
 
\item 使用tr命令“统一”字母大小写\\
小写 --> 大写:\verb?cat file | tr [a-z] [A-Z] > new_file?\\
大写 --> 小写:\verb?cat file | tr [A-Z] [a-z] > new_file?
 
\item 把文件中的数字0-9替换为a-j\\
\verb?cat file | tr [0-9] [a-j] > new_file?

\item 删除文件file中出现的“Snail”字符\\
\verb?cat file | tr -d "Snail" > new_file?  注意:这里,凡是在file文件中出现的'S','n','a','i','l'字符都会被删除!而不是紧紧删除出现的"Snail”字符串。
 
\item 删除文件file中出现的换行“\verb|\n|”、制表“\verb|\t|”字符\\
\verb?cat file | tr -d "\n\t" > new_file? 不可见字符都得用转义字符来表示的,这个都是统一的。
 
\item 删除“连续着的”重复字母,只保留第一个\\
\verb?cat file | tr -s [a-zA-Z] > new_file?
 
\item 删除空行\\
\verb?cat file | tr -s "\n" > new_file?
 
\item 删除Windows文件“造成”的“\verb|^M|”字符\\
\verb?cat file | tr -d "\r" > new_file?
或者
\verb?cat file | tr -s "\r" "\n" > new_file?
注意:这里-s后面是两个参数“\verb|\r|”和“\verb|\n|”,用后者替换前者
 
\item 用空格符\verb|\040|替换制表符\verb|\011|\\ 
\verb?cat file | tr -s "\011" "\040" > new_file?
 
\item 把路径变量中的冒号“:”,替换成换行符"\verb|\n|"\\
\verb?echo $PATH | tr -s ":" "\n"?
\end{enumerate}




\subsection{cut}
cut是一个选取命令,就是将一段数据经过分析,取出我们想要的。一般来说,选取信息通常是针对“行”来进行分析的,并不是整篇信息分析的。

其语法格式为:cut  [-bn] [file] 或 cut [-c] [file]  或  cut [-df] [file]

cut 命令从文件的每一行剪切字节、字符和字段并将这些字节、字符和字段写至标准输出。
如果不指定 File 参数,cut 命令将读取标准输入。必须指定 -b、-c 或 -f 标志之一。

主要参数
\begin{itemize}
\item
-b :以字节(bytes)为单位进行分割。这些字节位置将忽略多字节字符边界,除非也指定了 -n 标志。
\item
-c :以字符(characters)为单位进行分割。
\item
-d :自定义分隔符,默认为制表符。
\item
-f  :与-d一起使用,指定显示哪个区域(fields)。
\item
-n :取消分割多字节字符。仅和 -b 标志一起使用。如果字符的最后一个字节落在由 -b 标志的 List 参数指示的<br />范围之内,该字符将被写出;否则,该字符将被排除。
\end{itemize}

例如:
\begin{lstlisting}[language=sh]
cut -d' ' -f7-9 a.txt
\end{lstlisting}
表示以空格为分隔符剪切出a.txt的第7、8、9列。


\subsection{wc}
语法:

\verb*|wc [-clw] [--help] [--version] [文件……]|

参数:
\begin{itemize}
\item \verb|-c|或\verb|--bytes|或\verb|--chars| 只显示Bytes数。
\item \verb|-l|或\verb|--lines| 只显示列数。
\item \verb|-w|或\verb|--words| 只显示字数。
\item \verb|--help| 在线帮助
\item \verb|--version| 显示版本信息。
\end{itemize}

实例:

在默认的情况下,wc将计算指定文件的行数、字数,以及字节数。使用的命令为:

\verb|wc testfile|

运行结果例如:

\begin{lstlisting}[language=sh]
3 92 598 testfile       # testfile文件的行数为3、单词数92、字节数598 
\end{lstlisting}

也可以同时统计多个文件,如
\begin{lstlisting}[language=sh]
$ wc testfile testfile_1 testfile_2  #统计三个文件的信息  
3 92 598 testfile                    #第一个文件行数为3、单词数92、字节数598  
9 18 78 testfile_1                   #第二个文件的行数为9、单词数18、字节数78  
3 6 32 testfile_2                    #第三个文件的行数为3、单词数6、字节数32  
15 116 708 总用量                    #三个文件总共的行数为15、单词数116、字节数708 
\end{lstlisting}



\subsection{sed}
%http://www.cnblogs.com/dong008259/archive/2011/12/07/2279897.html
sed是一个很好的文件处理工具,本身是一个管道命令,主要是以行为单位进行处理,可以将数据行进行替换、删除、新增、选取等特定工作,下面先了解一下sed的用法

sed命令行格式为: \verb| sed [-nefri] ‘command’ 输入文本 |      

常用选项:
\begin{itemize}
\item  -n∶使用安静(silent)模式。在一般 sed 的用法中,所有来自 STDIN的资料一般都会被列出到萤幕上。但如果加上 -n 参数后,则只有经过sed 特殊处理的那一行(或者动作)才会被列出来。
\item  -e∶直接在指令列模式上进行 sed 的动作编辑;
\item  -f∶直接将 sed 的动作写在一个档案内, -f filename 则可以执行 filename 内的sed 动作;
\item  -r∶sed 的动作支援的是延伸型正规表示法的语法。(预设是基础正规表示法语法)
\item  -i∶直接修改读取的档案内容,而不是由萤幕输出。       
\end{itemize}

常用命令:
\begin{itemize}
\item  a   ∶新增, a 的后面可以接字串,而这些字串会在新的一行出现(目前的下一行)~
\item c   ∶取代, c 的后面可以接字串,这些字串可以取代 n1,n2 之间的行!
\item        d   ∶删除,因为是删除啊,所以 d 后面通常不接任何咚咚;
\item         i   ∶插入, i 的后面可以接字串,而这些字串会在新的一行出现(目前的上一行);
\item         p  ∶列印,亦即将某个选择的资料印出。通常 p 会与参数 sed -n 一起运作~
\item          s  ∶取代,可以直接进行取代的工作哩!通常这个 s 的动作可以搭配正规表示法!例如 1,20s/old/new/g 就是啦!
\end{itemize}

\begin{lstlisting}[language=sh]
举例:(假设我们有一文件名为ab)
     删除某行
     [root@localhost ruby] # sed '1d' ab              #删除第一行 
     [root@localhost ruby] # sed '$d' ab              #删除最后一行
     [root@localhost ruby] # sed '1,2d' ab           #删除第一行到第二行
     [root@localhost ruby] # sed '2,$d' ab           #删除第二行到最后一行

  显示某行
.    [root@localhost ruby] # sed -n '1p' ab           #显示第一行 
     [root@localhost ruby] # sed -n '$p' ab           #显示最后一行
     [root@localhost ruby] # sed -n '1,2p' ab        #显示第一行到第二行
     [root@localhost ruby] # sed -n '2,$p' ab        #显示第二行到最后一行

  使用模式进行查询
     [root@localhost ruby] # sed -n '/ruby/p' ab    #查询包括关键字ruby所在所有行
     [root@localhost ruby] # sed -n '/\$/p' ab        #查询包括关键字$所在所有行,使用反斜线\屏蔽特殊含义

  增加一行或多行字符串
     [root@localhost ruby]# cat ab
     Hello!
     ruby is me,welcome to my blog.
     end
     [root@localhost ruby] # sed '1a drink tea' ab  #第一行后增加字符串"drink tea"
     Hello!
     drink tea
     ruby is me,welcome to my blog. 
     end
     [root@localhost ruby] # sed '1,3a drink tea' ab #第一行到第三行后增加字符串"drink tea"
     Hello!
     drink tea
     ruby is me,welcome to my blog.
     drink tea
     end
     drink tea
     [root@localhost ruby] # sed '1a drink tea\nor coffee' ab   #第一行后增加多行,使用换行符\n
     Hello!
     drink tea
     or coffee
     ruby is me,welcome to my blog.
     end

  代替一行或多行
     [root@localhost ruby] # sed '1c Hi' ab                #第一行代替为Hi
     Hi
     ruby is me,welcome to my blog.
     end
     [root@localhost ruby] # sed '1,2c Hi' ab             #第一行到第二行代替为Hi
     Hi
     end

  替换一行中的某部分
  格式:sed 's/要替换的字符串/新的字符串/g'   (要替换的字符串可以用正则表达式)
     [root@localhost ruby] # sed -n '/ruby/p' ab | sed 's/ruby/bird/g'    #替换ruby为bird
   [root@localhost ruby] # sed -n '/ruby/p' ab | sed 's/ruby//g'        #删除ruby

     插入
     [root@localhost ruby] # sed -i '$a bye' ab         #在文件ab中最后一行直接输入"bye"
     [root@localhost ruby]# cat ab
     Hello!
     ruby is me,welcome to my blog.
     end
     bye

     删除匹配行

      sed -i '/匹配字符串/d'  filename  (注:若匹配字符串是变量,则需要“”,而不是‘’。记得好像是)

      替换匹配行中的某个字符串

      sed -i '/匹配字符串/s/替换源字符串/替换目标字符串/g' filename
\end{lstlisting}




\subsubsection{sed删除行首行尾空格和Tab}
(1)用 sed 实现删除行首空格:\verb|sed 's/^[ \t]*//g'|

说明: 
\begin{itemize}
\item 第一个/的左边是s表示替换,即将空格替换为空;
\item 第一个/的右边是表示后面的以xx开头 中括号表示“或”,空格或Tab中的任意一种。这是正则表达式的规范。 中括号右边是*,表示一个或多个;
\item 第二个和第三个/中间没有东西,表示空 g表示替换原来buffer中的,sed在处理字符串的时候并不对源文件进行直接处理,先创建一个buffer,但是加g表示对原buffer进行替换 整体的意思是:用空字符去替换一个或多个用空格或tab开头的本体字符串 。
\end{itemize}

(2)用 sed 实现删除行末空格:\verb|sed 's/[ \t]*$//g' |

和上面稍微有些不同是前面删除了\verb|^|符,在后面加上了美元符,这表示以xx结尾的字符串为对象。 
但是要注意在KSH中,Tab并不是\verb|\t|而是直接打入一个Tab就可以了。

(3)删除行首空格、多余空格已经空行

\verb!cat 2dtemp|sed 's/^[ \t]*//g'|tr -s '\n'|tr -s ' ' > 2d.ods!


\section{输出命令}
\subsection{echo}
echo命令用于在屏幕上打印出指定的字符串。命令格式:

\verb|echo arg|

可以使用echo实现更复杂的输出格式控制。

(1)显示转义字符
\begin{lstlisting}[language=sh]
echo "\"It is a test\""
\end{lstlisting}
运行结果是:
\verb|"It is a test"|

其中双引号也可以省略。

(2)显示变量
\begin{lstlisting}[language=sh]
name="OK"
echo "$name It is a test"
\end{lstlisting}
运行结果是:
\verb|OK It is a test|

同样双引号也可以省略。

如果变量与其它字符相连的话,需要使用大括号(\{ \}):
\begin{lstlisting}[language=sh]
mouth=8
echo "${mouth}-1-2009"
\end{lstlisting}
运行结果是:
\verb|8-1-2009|

(3)显示换行
\begin{lstlisting}[language=sh]
echo "OK!\n"
echo "It is a test"
\end{lstlisting}
运行结果是:
\begin{verbatim}
OK!
It is a test
\end{verbatim}

(4)显示不换行
\begin{lstlisting}[language=sh]
echo "OK!\c"
echo "It is a test"
\end{lstlisting}
运行结果是:
\begin{verbatim}
OK!It si a test
\end{verbatim}

(5)显示结果重定向至文件
\begin{lstlisting}[language=sh]
echo "It is a test" > myfile
\end{lstlisting}

(6)原样输出字符串

若需要原样输出字符串(不进行转义),请使用单引号。例如:
\begin{lstlisting}[language=sh]
echo '$name\"'
\end{lstlisting}

(7)显示命令执行结果
\begin{lstlisting}[language=sh]
echo `date`
\end{lstlisting}


\subsection{printf}
printf 命令用于格式化输出, 是echo命令的增强版。它是C语言printf()库函数的一个有限的变形,并且在语法上有些不同。printf 由 POSIX 标准所定义,移植性要比 echo 好。

如同 echo 命令,printf 命令也可以输出简单的字符串:
\begin{lstlisting}[language=sh]
$printf "Hello, Shell\n"
Hello, Shell
$
\end{lstlisting}
printf 不像 echo 那样会自动换行,必须显式添加换行符(\verb|\|n)。

printf 命令的语法:

\verb|printf  format-string  [arguments...]|

format-string 为格式控制字符串,arguments 为参数列表。

printf命令与C语言中printf()功能和用法类似,不同的是:
\begin{itemize}
\item printf 命令不用加括号;
\item format-string 可以没有引号,但最好加上,单引号双引号均可;
\item 参数多于格式控制符(\%)时,format-string 可以重用,可以将所有参数都转换;
\item arguments 使用空格分隔,不用逗号。
\end{itemize}
请看下面的例子:
\begin{lstlisting}[language=sh]
# format-string为双引号
$ printf "%d %s\n" 1 "abc"
1 abc
# 单引号与双引号效果一样 
$ printf '%d %s\n' 1 "abc" 
1 abc
# 没有引号也可以输出
$ printf %s abcdef
abcdef
# 格式只指定了一个参数,但多出的参数仍然会按照该格式输出,format-string 被重用
$ printf %s abc def
abcdef
$ printf "%s\n" abc def
abc
def
$ printf "%s %s %s\n" a b c d e f g h i j
a b c
d e f
g h i
j
# 如果没有 arguments,那么 %s 用NULL代替,%d 用 0 代替
$ printf "%s and %d \n" 
and 0
# 如果以 %d 的格式来显示字符串,那么会有警告,提示无效的数字,此时默认置为 0
$ printf "The first program always prints'%s,%d\n'" Hello Shell
-bash: printf: Shell: invalid number
The first program always prints 'Hello,0'
$
\end{lstlisting}
注意,根据POSIX标准,浮点格式\%e、\%E、\%f、\%g与\%G是“不需要被支持”。这是因为awk支持浮点预算,且有它自己的printf语句。这样Shell程序中需要将浮点数值进行格式化的打印时,可使用小型的awk程序实现。然而,内建于bash、ksh93和zsh中的printf命令都支持浮点格式。



\section{系统配置管理命令}
\subsection{date}
语法:
\begin{lstlisting}[language=sh]
date [-u] [-d datestr] [-s datestr] [--utc] [--universal] [--date=datestr] [--set=datestr] [--help] [--version] [+FORMAT] [MMDDhhmm[[CC]YY][.ss]]
\end{lstlisting}

参数说明:
\begin{itemize}
\item -d datestr : 显示 datestr 中所设定的时间 (非系统时间)
\item --help : 显示辅助讯息
\item -s datestr : 将系统时间设为 datestr 中所设定的时间
\item -u : 显示目前的格林威治时间
\item --version : 显示版本编号
\end{itemize}

显示格式:
\\
显示格式设定为一个加号后接数个标记,其中可用的标记列表如下
\\
时间方面:
\begin{itemize}
\item \% : 印出 \%
\item\%n : 下一行
\item\%t : 跳格
\item\%H : 小时(00..23)
\item\%I : 小时(01..12)
\item\%k : 小时(0..23)
\item\%l : 小时(1..12)
\item\%M : 分钟(00..59)
\item\%p : 显示本地 AM 或 PM
\item\%r : 直接显示时间 (12 小时制,格式为 hh:mm:ss [AP]M)
\item\%s : 从 1970 年 1 月 1 日 00:00:00 UTC 到目前为止的秒数
\item\%S : 秒(00..61)
\item\%T : 直接显示时间 (24 小时制)
\item\%X : 相当于 \%H:\%M:\%S
\item\%Z : 显示时区
\end{itemize}

日期方面:
\begin{itemize}
\item\%a : 星期几 (Sun..Sat)
\item\%A : 星期几 (Sunday..Saturday)
\item\%b : 月份 (Jan..Dec)
\item\%B : 月份 (January..December)
\item\%c : 直接显示日期与时间
\item\%d : 日 (01..31)
\item\%D : 直接显示日期 (mm/dd/yy)
\item\%h : 同 \%b
\item\%j : 一年中的第几天 (001..366)
\item\%m : 月份 (01..12)
\item\%U : 一年中的第几周 (00..53) (以 Sunday 为一周的第一天的情形)
\item\%w : 一周中的第几天 (0..6)
\item\%W : 一年中的第几周 (00..53) (以 Monday 为一周的第一天的情形)
\item\%x : 直接显示日期 (mm/dd/yy)
\item\%y : 年份的最后两位数字 (00.99)
\item\%Y : 完整年份 (0000..9999)
\end{itemize}

若是不以加号作为开头,则表示要设定时间,而时间格式为 MMDDhhmm[[CC]YY][.ss],其中 MM 为月份,DD 为日,hh 为小时,mm 为分钟,CC 为年份前两位数字,YY 为年份后两位数字,ss 为秒数。使用权限:所有使用者。

当您不希望出现无意义的 0 时(比如说 1999/03/07),则可以在标记中插入 - 符号,比如说 date '+\%-H:\%-M:\%-S' 会把时分秒中无意义的 0 给去掉,像是原本的 08:09:04 会变为 8:9:4。另外,只有取得权限者(比如说 root)才能设定系统时间。

当您以 root 身分更改了系统时间之后,请记得以 clock -w 来将系统时间写入 CMOS 中,这样下次重新开机时系统时间才会持续抱持最新的正确值。

实例:
\begin{lstlisting}[language=sh]
# date
三 5月 12 14:08:12 CST 2010
# date '+%c' 
2010年05月12日 星期三 14时09分02秒
# date '+%D' //显示完整的时间
05/12/10
# date '+%x' //显示数字日期,年份两位数表示
2010年05月12日
# date '+%T' //显示日期,年份用四位数表示
14:09:31
# date '+%X' //显示24小时的格式
14时09分39秒
# date '+usr_time: $1:%M %P -hey'
usr_time: $1:16 下午 -hey
\end{lstlisting}



\subsection{source}
\subsection{export}



\subsection{ulimit}
想象一个状况:我的 Linux 主机里面同时登陆了十个人,这十个人不知怎么搞的, 同时开启了 100 个文件,每个文件的大小约 10MBytes ,请问一下, 我的 Linux 主机的内存要有多大才够? 10*100*10 = 10000 MBytes = 10GBytes ... 老天爷,这样,系统不挂点才有鬼哩!为了要预防这个情况的发生,所以我们的 bash 是可以『限制用户的某些系统资源』的,包括可以开启的文件数量, 可以使用的 CPU 时间,可以使用的内存总量等等。如何配置?用 ulimit 吧!

\verb|ulimit [-SHacdfltu] [配额]|

选项与参数:
-H  :hard limit ,严格的配置,必定不能超过这个配置的数值;
-S  :soft limit ,警告的配置,可以超过这个配置值,但是若超过则有警告信息。
      在配置上,通常 soft 会比 hard 小,举例来说,soft 可配置为 80 而 hard 
      配置为 100,那么你可以使用到 90 (因为没有超过 100),但介于 80~100 之间时,
      系统会有警告信息通知你!
-a  :后面不接任何选项与参数,可列出所有的限制额度;
-c  :当某些程序发生错误时,系统可能会将该程序在内存中的信息写成文件(除错用),
      这种文件就被称为核心文件(core file)。此为限制每个核心文件的最大容量。
-f  :此 shell 可以创建的最大文件容量(一般可能配置为 2GB)单位为 Kbytes
-d  :程序可使用的最大断裂内存(segment)容量;
-l  :可用于锁定 (lock) 的内存量
-t  :可使用的最大 CPU 时间 (单位为秒)
-u  :单一用户可以使用的最大程序(process)数量。


\subsection{history}



\subsection{basename}
首先使用 --help 参数查看一下。basename命令参数很少,很容易掌握。
\begin{verbatim}
Usage: basename NAME [SUFFIX]
  or:  basename OPTION... NAME...
Print NAME with any leading directory components removed.
If specified, also remove a trailing SUFFIX.

Mandatory arguments to long options are mandatory for short options too.
  -a, --multiple       support multiple arguments and treat each as a NAME
  -s, --suffix=SUFFIX  remove a trailing SUFFIX
  -z, --zero           separate output with NUL rather than newline
      --help     display this help and exit
      --version  output version information and exit

Examples:
  basename /usr/bin/sort          -> "sort"
  basename include/stdio.h .h     -> "stdio"
  basename -s .h include/stdio.h  -> "stdio"
  basename -a any/str1 any/str2   -> "str1" followed by "str2"
\end{verbatim}




\subsection{passwd——更改密码命令}
passwd是“password”的缩写\footnote{类似的一个缩写命令pwd已经被print working directory用了,所以不要记错了}。在输入密码时,字符是不回显的,这可以防止他人偷看到新密码的长度,进一步降低被破解的可能性\footnote{可以体会到Linux系统的高度安全性吧}。


\subsection{iconv——文本编码转换命令}
iconv -f encoding [-t encoding] [inputfile]... 转换给出文件的编码从一种到另外一种。

\begin{center}
\begin{tabular}{lc}
\hline
 \bf{输入/输出格式规范:}           &         \\
 -f, --from-code=NAME     &    原始文本编码\\
 -t, --to-code=NAME           &    输出编码  \\
\hline
\bf{信息:}                                &         \\
 -l, --list                                      &    列出所有已知编码字符集\\
\hline
\bf{输出控制:}                      &     \\
 -c                                                 &忽略输出中的无效字符\\
 -o, --output=FILE                & 输出文件\\
 -s, --silent                                & suppress warnings\\
     --verbose                            & 打印进程信息\\
 -?, --help                                  & 显示此帮助列表\\
     --usag                                   &提供简短的使用信息\\
 -V, --version                          &显示程序版\\
\end{tabular}\\
\end{center}
对长选项必须或可选的参数同样适用于相应的短选项。

\textbf{示例:}\\
iconv -f=gbk -t=utf8 /home/phileas/ANSI-CJKmain.tex~-o /home/phileas/utf8-cjkmain.tex

其中:-f=gbk -t=utf8意思是将gbk编码转换成utf8编码;/home/phileas/ANSI-CJKmain.tex是需要转换文件的全路径;-o 是输出文件;/home/phileas/utf8-cjkmain.tex是输出文件的全路径。如果觉得写路径麻烦,那就直接用命令cd到该文件所处的文件夹下,就不用输入路径了。


\subsection{ulimit命令}
ulimit用于shell启动进程所占用的资源,是shell内建命令,语法格式:

ulimit [-acdfHlmnpsStvw] [size]

参数介绍:
\begin{itemize}
\item -a 显示当前所有的资源限制.
\item -c size:设置core文件的最大值.单位:blocks
\item -d size:设置数据段的最大值.单位:kbytes
\item -f size:设置创建文件的最大值.单位:blocks
\item -l size:设置在内存中锁定进程的最大值.单位:kbytes
\item -m size:设置可以使用的常驻内存的最大值.单位:kbytes
\item -n size:设置内核可以同时打开的文件描述符的最大值.单位:n
\item -p size:设置管道缓冲区的最大值.单位:kbytes
\item -s size:设置堆栈的最大值.单位:kbytes
\item -t size:设置CPU使用时间的最大上限.单位:seconds
\item -v size:设置虚拟内存的最大值.单位:kbytes
\item -u number:设置用户最大进程数 (max user processes)
\item -H 设置硬件资源限制.
\item -S 设置软件资源限制.
\end{itemize}

在Linux下写程序的时候,如果程序比较大,经常会遇到“段错误” (segmentation fault)这样的问题,这主要就是由于Linux系统初始的堆栈大小(stack size)太小的缘故,一般为10M。我一般把stack size设置成256M,这样就没有段错误了!命令为:

ulimit   -s 262140

如果要系统自动记住这个配置,就编辑/etc/profile文件,在 “\verb|ulimit -S -c 0 > /dev/null 2>&1|”行下,添加“ulimit   -s 262140”,保存重启系统就可以了。

调整ulimit的stack size 参数调整为unlimited 无限,使用ulimit -s unlimited时只能在当时的shell见效,重开一个shell就失效了。于是得在/etc/profile 的最后面添加ulimit -s unlimited 就可以了,source /etc/profile使修改文件生效。





\section{Makefile}
Makefile 的作用就是——自动化编译,一旦写好,只需要一个 make 命令来解析 Makefile,执行 Makefile 中描述的操作,整个工程就能自动编译,无论这个工程拥有多少个源代码文件。 Makefile 定义了一系列的规则,来指定哪些文件需要先编译,哪些文件需要后编译,哪些文件需要重新编译,甚至进行更复杂的功能操作,因为 Makefile 就像一个 shell 脚本一样,其中也可以执行操作系统的命令。

Makefile 是当前 Linux 下绝大多数软件的工程管理工具。而且它几乎已经成为了一个事实上的标准。
即便不具备编写 Makefile 的技能,那也需要掌握 make 的使用方法。因为还有很大一部分需要的、或者感兴趣的软件都是采用源代码方式发行的。即便 rpm、deb等软件打包方案在 Linux 世界已经大行其道,但是如果要使用这些软件的最新版本,基本上脱离不了使用 make 这个工具通过源代码来创建它。即便那些采用 Python、Perl 等解释语言开发的软件,绝大多数的情况也是利用 Makefile 作为安装脚本的。当然,更进一步说,有了一个很了不起的创意,并且准备分享给全世界,那么在 Linux 上就更需要掌握 make 乃至 Makefile 的编写技巧。

实际上 Makefile 是一类工程管理工具的工程描述文件的默认名称,让人头疼的是,这个文件名还不是唯一的。可以是 Makefile,可以是 makefile\footnote{注意 Linux 的文件名区分大小写},甚至是 GNUmakefile。Makefile 是 make 工程管理工具的工程描述文件, make 是类 UNIX 世界的一个传统工程管理工具,是它们文化的一部分,可以称作 Makefile 文化。传统 UNIX 的主要分支,比如 Solaris、HP-UNIX、AIX 等采用一套 Makefile 体系;BSD系列如: FreeBSD、OpenBSD、NetBSD 等采用一套 Makefile 体系;Linux 采用另一套 Makefile 体系。 而且即便属于同一体系,在某些方面都会显得大相径庭。

在 Linux 中采用的是 GNU make 这一体系,属于 GNU 计划中的一部分,可以说是所有体系中功能最为丰富、兼容性最好的一个 Makefile 文化的传承品。GNU make 已经是当前事实上的“标准”了,就好像它说的是标准普通话,其他  make 说的都是方言。



\subsection{make命令的基本用法}
代码变成可执行文件,叫做编译(compile);先编译这个,还是先编译那个(即编译的安排),叫做构建(build)。Make是最常用的构建工具,诞生于1977年,主要用于C语言的项目。

make 是 Linux 系统下的一个工具,在任意目录中执行 make 都可以。 make 会在执行路径中搜索 Makefile 文件。Makefile 并不是唯一的选择,还可以是 makefile 和 GNUmakefile。如果一个路径中同时存在它们三者,make会选择一个。make的选择顺序是 GNUmakefile、makefile 和 Makefile。只要选中了一个就不再理会其他的了。

选项:






make clean

清除上次的make命令所产生的object文件(后缀为“.o”的文件)及可执行文件。

make distclean

类似make clean,但同时也将configure生成的文件全部删除掉,包括Makefile。
make clean仅仅是清除之前编译的可执行文件及配置文件,而make distclean要清除所有生成的文件。

make install

将编译成功的可执行文件安装到系统目录中,一般为/usr/local/bin目录。

make dist

产生发布软件包文件(即distribution package)。这个命令将会将可执行文件及相关文件打包成一个tar.gz压缩的文件用来作为发布软件的软件包。

它会在当前目录下生成一个名字类似“PACKAGE-VERSION.tar.gz”的文件。PACKAGE和VERSION,是我们在configure.in中定义的\verb*|AM_INIT_AUTOMAKE(PACKAGE, VERSION)|。

make distcheck

生成发布软件包并对其进行测试检查,以确定发布包的正确性。这个操作将自动把压缩包文件解开,然后执行configure命令,并且执行make,来确认编译不出现错误,最后提示你软件包已经准备好,可以发布了。





























