
\chapter*{前言}
\addcontentsline{toc}{chapter}{前言}
\section*{文档定位}
\addcontentsline{toc}{section}{文档定位}
\begin{enumerate}
\item 自己使用过程的积累,大部分转载自网上,我会尽量写明转载地址,难免有些忘了,还请见谅。
\item 对一个程序的应用不熟,看它的说明文档来的最快!
\end{enumerate}

科研心得


\section*{前言一}
\addcontentsline{toc}{section}{前言一}
    Ubuntu 更新很快,每年都发布两个版本,4月一个,10月一个,很多技术技巧很快会成为历史\footnote{但是 Linux 的那些基本的东西可以说是不会变的},注意技术技巧的时间是很有必要的。

    我使用 Ubuntu 的过程可谓是曲曲折折、磕磕碰碰、一步一步走来。最开始接触到 Ubuntu 是08年在学校 BBS 上看到相关的帖子,被 Ubuntu 的 3D 特效和开放源代码所吸引。最开始是怎么装的 Ubuntu 记得不是很清楚,好像是下载光盘镜像刻录在光盘上,用光盘装的,装的是双系统。遇到的最大问题是,上外网的问题,学校上外网是用客户端的,按流量计费。但是在 Ubuntu 下搞定这个客户端一直没有很好解决。还有一个没有很好解决的是显卡驱动的问题。而最最重要的是 Windows  XP 下的很多软件在 Ubuntu 下没有很好的对应版本,如 QQ 和 银行网盾。学校玩 Ubuntu 的人也很少。于是虽说是双系统,但大部分还是在用 XP, Ubuntu 成为一个摆设。后来因为占用硬盘,干脆格掉了。这是第一个时期。过了一段时间,手又痒了,这次学会了用 U盘装,还是双系统。虽然Ubuntu 下有网页登录客户端,但始终没有摆脱 Windows,Ubuntu 再次成为摆设。没有摆脱Windows,Ubuntu 的使用始终是很难发展的。第三个时期,也就是现在,可以总算对 Windows 说再见了。毕业了,没有了客户端,上网直接拨号,上网问题不存在了;电脑换了,Ubuntu 也发展了很多,显卡驱动问题不存在了;WebQQ 也完善了许多,QQ 问题不在了;Ubuntu 有了专门的软件商店,里面有很多免费的开放源代码的软件,再加上虚拟机 Virtualbox 功能强大,我在 Ubuntu 下设了个虚拟机装了 XP 来应对一些目前没法替代的软件,其实也就是网购、网银那些了,这样软件的问题可以说也不存在了。时至今日,我终于才迈过 Ubuntu 的门槛,而这花了我四年时间。这个时间真是有点长,有各种因素,最本质的还是自己真正花在这上面的时间太少。欣慰的是,我终究没有放弃这。
    
 这个使用记是伴随 Ubuntu 的更新一起前进的,最开始是基于11.10,现在基于12.04 。有些技巧会过时,有些基本的却不会变,这得自己好好斟酌体会。
 
\rightline{——20120505\qquad}



\section*{前言二}
\addcontentsline{toc}{section}{前言二}
这是这份笔记结构的第一次重大调整,原名叫《Ubuntu 使用记》,主要是我使用 Ubuntu 过程中的记录。最开始写这份笔记的时候是2012年,现在已经是2014年,近两年了。为什么现在才考虑进行结构调整并改名为《计算机的使用》,或许是现在对计算机了解的更深入,使用的更多吧。本笔记的主题将是自己对计算机所有相关的学习记录,包括系统 Linux 和其中一个发行版 Ubuntu 的使用,也有脚本语言 Shell 的学习,也有 Fortran、C、C++等语言的学习。

\rightline{——20140307\qquad}

定位为纯粹的个人的计算机学习记录,记的都是自己想记下的东西。

\rightline{——20140719\qquad}



\section*{前言三}
\addcontentsline{toc}{section}{前言三}
这是这份文档的第二次重大调整,原名叫《计算机的使用》,现改名叫《科研中计算机的使用》。
缩小范围,针对科研服务的。

 








\section*{Ubuntu 使用大事件}
\addcontentsline{toc}{section}{Ubuntu 使用大事件}
\subsection*{2014.7.30——Ubuntu和Win7双系统安装最终解决之道}
\addcontentsline{toc}{subsection}{Ubuntu和Win7双系统安装最终解决之道}
%\centerline{{\sanhao Ubuntu和Win7双系统安装最终解决之道}}
这两天,真的是备受煎熬。由于历史原因,没想到自己对 Linux 系统的需求性越来越强,当时给笔记本安装 Ubuntu 时分配的硬盘空间太小了,不够用,要重装。说实话,自己是很不愿意重装的,毕竟用了很久了,很多东西要重新安装,配置,很麻烦的。一重装,问题就来了,这也确定了我接下来两天煎熬的日子。安装 Ubuntu 的时候,不能识别已经有的 Win7。怀疑系统版本问题,怀疑启动盘没制作好,怀疑电脑硬盘分区问题。最后已经存在的 Win7 也重装了,电脑系统算是彻底重装了。Win7 系统版本不好,装了不满意,网上下载,那天却莫名其妙的断网。重装了几次 Win7,硬盘分区也试了很多次,安装 Ubuntu 的时候总是检测不到已有的 Win7。很烦躁,不知道自己能不能搞定,甚至质疑自己当初重装系统的决定。不过,这个决定是正确的,这条路必须走,问题必须拿下。问题到了第二天有了转机,终于在网上搜到问题的解释。迷雾渐渐消散,我也似乎领会到了解决之道。

自己分区安装 Ubuntu,然后在 Win7 下用软件 EasyBCD 建立引导。 Linux 文件系统清晰明了,很喜欢,我只分三个区就行了。 1、/ boot 分区, 包含了操作系统的内核和在启动系统过程中所要用到的文件,一般 200 MB 就行了;2、 /swap 交换分区,物理内存大于512MB时分配与物理内存等容量,也是一种文件系统,它的作用是作为 Linux 的虚拟内存。 在 Windows 下, 虚拟内存是一个文件, pagefile.sys,而在 Linux 下,虚拟内存需要使用独立分区; 3、/分区,剩下的容量都用于根分区了,觉得没必要细分,细分反而麻烦。 至于如何在 Win7 下用软件 EasyBCD 建立引导,网上资料较多,搜索便是。

安装方法是正确的但也不是很顺利,安装过程不太正常,试了两次都是这样。怀疑制作安装盘的U盘出问题了。不巧,没其他U盘了,又开始借U盘,也不是很顺利,跟技术没关系就不讲了。果真是U盘的问题,期待许久的界面终于在第二天晚上快要回宿舍的时间见到了。这次折腾,不是没又收获,其实还挺大的。对Ubuntu的安装有了深入的了解,可以说学会了终极安装的方法。在这个过程中,也发现了新的东西。只是希望,以后折腾是在风险可控的范围内尝试。虽然这次数据备份了,但是我只有这台笔记本,安装不上,那真是麻烦了。

失败确实是成功之母。我真的是在屡屡失败之中成长起来的。心态也要锻炼起来,不要起伏波动太大,从容应对。

no zuo no die。 就在写这个的过程中,我又折腾了。在试 nvidia 的显卡驱动,结果图形界面进不去了。好在还有命令行界面,多年的折腾经验告诉我,镇定,要镇定。果断网上搜索,一堆看法。后来想到,卸载nivdia驱动就行了,于是敲入 \verb|sudo apt-get --purge remove nvidia-*| 命令。然后,图形界面又回来了。果断还是用系统默认的显卡驱动吧。

做事之前,先做风险评估。尝试固然很好,但要做风险评估。



\subsection*{2014.9.11——双显卡驱动最终解决之道}
\addcontentsline{toc}{subsection}{双显卡驱动最终解决之道}
%\centerline{{\sanhao 双显卡驱动最终解决之道}}
Ubuntu 的显卡驱动问题一直都是很头疼的问题。这次不是手贱了。一次更新后,开机在5个点亮完就没反应了。进入修复模式,偶然瞥到“load fallback graphics devices [fail]”,不用说又是显卡驱动问题。其实没这个提示,出现这个问题也大概猜到是显卡驱动的问题。 回想到上次的经历,这次问题的处理心里大概有个底。首先想办法进入命令行模式。以往版本的系统启动时都有这个选项的,14.04版本隐藏了。本来想想办法调出这个模式的引导的。但在这之前抱着运气的态度再次尝试了修复模式。在修复模式下,选择进入低画质模式,虽然成功登陆,但是桌面空白,只有鼠标,可以理解,本来就是显卡驱动出问题了嘛。突然想起进入命令行模式的快捷键,Ctrl+Alt+F1,果然进入了。然后就是按照上次的经历,输入命令,移除所有关于 nvidia 的官方驱动,即:\\
\verb|apt-get --purge remove nvidia-*|

成功了!图形界面进去了。可是问题没完,ubuntu 的开源显卡驱动,不太好。首先,显示面只占了方形区域,两边空白,鼠标还有闪烁,很不好用。安装 nivida 官方驱动又出问题,怎么办? 我想先查看下我的电脑的显卡信息吧,\verb[lspci |grep VGA[,赫然显示:

00:02.0 VGA compatible controller: Intel Corporation 2nd Generation Core Processor Family Integrated Graphics Controller (rev 09)

01:00.0 VGA compatible controller: NVIDIA Corporation GF108M [GeForce GT 630M] (rev ff)

我的电脑是双显卡,立即想起关于双显卡驱动的 Bumblebee Project。 打开主页 \url{http://www.bumblebee-project.org}, 按照说明,安装Bumblebee,重启,无需再额外装 nvidia 驱动了,世界平静了。


\subsection*{2014.10.15——又进不了系统了}
\addcontentsline{toc}{subsection}{又进不了系统了}
%\centerline{{\sanhao 又进不了系统了}}
今天弹出软件更新,看着就更新了下。可是没想到,在下载flash插件的时候一直卡着下载不下来。我下意识的就知道问题要来了,但也没办法只能强关。重启后,果然出了问题,无线网卡打不开。我知道是之前更新的问题,于是想着重新更新下。可是还是卡在flash插件的更新那了。只能再次强关,这次就没那么幸运,系统果断进不了了。网上查询了下,貌似用 Ubuntu  启动优盘启动,然后在运行命令修护。虽然操作看着还会,只是挺麻烦的。后来想着开机时有修护模式,就试着进入了修护模式。但是第一次进入修护模式还是老问题。后来第二次进入修护模式,选了较低版本的内核,成功进入了修护模式。运行了软件包修护选项,经过一系列的修护后,终于又进入系统了。后来重启也正常了第一次进入修护模式没有成功,大概是最新的内核在更新的时候出了问题吧,所以选用久了内核成功了。Linux 系统真的好折腾人啊,不敢再更新了。

也别一朝被蛇咬,十年怕井绳。以后更新的时候,注意看看更新文件。那个可能引起麻烦的,Ubuntu 用了这么多年了,心里也有个数。不要因为困难而停止进步。



\section*{学习及参考资料}
\addcontentsline{toc}{section}{学习及参考资料}
\textbf{参考书}
\begin{itemize}
\item 《鸟哥的Linux私房菜:基础学习篇》鸟哥,王世江;
\item 《Linux就是这个范儿》赵鑫磊,张洁;
\item 《Linux命令行与Shell脚本编程大全》Richard Blum,Christine Bresnahan,武海峰(译);
\item 《Unix\&Linux大学教程》Harley Hahn,张杰良(译)
\end{itemize}

\textbf{不错的学习网站}
\begin{itemize}
\item \url{http://www.92csz.com/study/linux/}

\item \url{http://vbird.dic.ksu.edu.tw/}

\item Python学习:\url{http://www.w3cschool.cc/python/python-tutorial.html}
\end{itemize}




\chapter{科研入门技能及心得}
\section{一些该注册的}
gmail: 至少两个。一个注册用,一个通讯用。真名。

research gate

google scholar

ocrc

\section{文献检索与管理}
google scholar 检索

zotero 管理

\section{Pdf批量全文搜索}
Recoll

\url{http://www.linuxdown.net/install/faq/20160318_how_linux_5065.html}



\section{程序和文档版本管理——Git\&Github}
以我们的量子耗散动力学、光谱计算为例
\begin{enumerate}
\item 找个合适的模型作为标准模型;
\item 只要新增内容,就新复制一个,说明改进;
\item 所有新功能先在标准模型上实现,再应用;
\item 结合git版本控制。
\end{enumerate}


参考:bilibili 搜索git,如\underline{尚硅谷官方}、庄七。

Git基本概念
\begin{itemize}
\item 工作区:写代码
\item 暂存区:临时存储
\item 本地库:历史版本
\end{itemize}

分支管理

\subsection{初始操作}
\begin{itemize}
\item[(1)] 本地库初始化(建立本地库)
\begin{itemize}
\item git init\\
进入项目目录,在命令行输入 git init,会产生一个.git的子目录,存放的是本地库相关的目录和文件,不要删除和胡乱修改。
\end{itemize}

\item[(2)] 设置签名
\begin{itemize}
\item git config user.name XX
\item git config user.email XX
\end{itemize}

\item [(3)] 配置
\begin{itemize}
\item \verb|git config --global core.editor  "vim"| : 使用Vim作为编辑器
\end{itemize}

\end{itemize}



\subsection{提交操作}
\begin{enumerate}
\item git status: 查看状态
\item \verb|git add <file>|: 添加追踪文件到缓存区 / \verb|git rm --cached <file>| :移除缓存区追踪文件
\item \verb|git commit <file>|:提交文件到库
\item \verb|git commit -a |:直接将所有修改文件提交到库(也可以分成两步,先git add更新缓存区,再git commit提交到库)
\item \verb|git commit -m ``xxx'' <file>|:跳过消息编辑器步骤,直接提交“xxx”更新记录
\end{enumerate}



\subsection{版本穿梭操作}
\begin{enumerate}
\item \verb|git log|:历史记录查询\\
精炼形式:\verb|git reflog| 

\item \verb|git reset --hard <index>|:历史版本选择\\
注意,还有其他两个参数\verb|--soft|、\verb|--mixed|。基于这个功能,可以实现删除文件并找回,但前提是文件存在时的状态必须提交到了本地库。
\end{enumerate}



\subsection{比较文件差异}
\begin{enumerate}
\item git diff <file> :将工作区中的文件和暂存区进行比较
\item git diff <本地库中某个历史版本> <file>:将工作区中的文件和本地库中历史文件进行比较
\end{enumerate}



\subsection{分支管理}
\begin{enumerate}
\item \verb|git branch -v |:查看分支
\item \verb|git branch <name>| :创建分支
\item \verb|git checkout <name>|:切换分支
\item 合并分支
\begin{itemize}
\item 切换到要合并到的分支
\item \verb|git merge <branch name>|:
\end{itemize}
\item 解决合并冲突
\begin{itemize}
\item  编辑文件,删除特殊符号
\item 把文件修改到满意的程度,保存退出
\item \verb|git commit -m | ``日志信息''。注意此时不能带文件名。
\end{itemize}
\end{enumerate}



\subsection{远程仓库Github}
\begin{itemize}
\item[(1)] 创建远程仓库:略!

\item[(2)] 推送到远程仓库
\begin{itemize}
\item \verb|git remote -v|:查看远程库
\item \verb|git remote add <name> <https://address> |:添加远程库
\item \verb|git push <name> <branch>|:推送分支
\end{itemize}

\item [(3)] ssh免密推送
\end{itemize}
